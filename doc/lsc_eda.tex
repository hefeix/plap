\documentclass[letterpaper]{article}
\usepackage{epsfig,algpseudocode,algorithm,amsmath}
\setlength{\pdfpagewidth}{8.5in}
\setlength{\pdfpageheight}{11in}
\topmargin 4.5pc
\advance\topmargin by -1.1in
\textheight 660pt
%\makeatletter
%\def\NAT@parse{\typeout{This is a fake Natbib command to fool Hyperref.}}
%\makeatother
%\usepackage[hyperfootnotes=false]{hyperref}
\usepackage{hyperref}
\hypersetup{
  letterpaper,
  pdfpagemode=none,
  pdftitle=
  {Generalized Schemata, Iterated Descent, and Estimation of Distributions},
  pdfauthor={Moshe Looks},
}
\DeclareMathOperator*{\argmax}{argmax}
\begin{document}
\sloppy
\title
{Generalized Schemata, Iterated Descent, and Estimation of Distributions}
\author{
  Moshe Looks
  \href{mailto:moshe@metacog.org}{\emph{madscience@google.com}}
}
\maketitle

\begin{abstract}
  I propose a new class of estimation-of-distribution algorithms. The
  theoretical basis is hierarchical problem decomposition into generalized
  schemata (building-blocks). The search procedure is based on focused random
  walks, Metropolis transitions, and iterated descent. The approach to
  probabilistic model-building is inspired by the building-block hill-climber
  and the hierarchical Bayesian optimization algorithm.
\end{abstract}

\section{Theoretical Basis}

Estimation-of-distribution algorithms (EDAs) are optimizers that maintain and
sample from a distribution of ``good'' solutions. It is important to note that
the character of this distribution - which novel solutions we will tend to
sample from it - depends crucially on what classes of subsolution features, or
\emph{schemata}, we deem useful a priori. In other words, the same data,
modeled with different classes of schemata, will yield different
generalizations. The majority of existing EDAs construct their models in based
on Holland's original schema formalism~\cite{Holland,Goldberg}.

The fundamental formalism of genetic algorithms is John Holland's schema: a
string of length $n$ drawn from the alphabet $\{?,0,1\}$, in the context of an
optimization problem over the domain $\{0,1\}^n$. For $n=4$, for example, the
schema $0?1?$ \emph{matches} the four solutions $0010$, $0011$, $0110$, and
$0111$ (these solutions are said to \emph{contain} the schema $0?1?$). It
\emph{competes} with the schemata $0?0?$, $1?0?$, and $1?1?$. It is mutually
exclusive to, or \emph{disrupts} all of the schemata that contain a $1$ in
first position or a $0$ in the third position ($1???$, $0001$, etc.).

The Bayesian Optimization Algorithm (BOA)~\cite{BOA} represents problem
decompositions for $n$-bit optimization problems as Bayesian Networks with one
variable per bit. What does the network structure actually \emph{mean}, in
terms of the individual bits in the optimization problem? Lets first consider a
network with no edges. The $i$th node in the network ($x_i$, where $0 \leq i <
n$) may be understood to model the competition between the schemata
$?^{i-1}0?^{n-i}$ and $?^{i-1}1?^{n-i}$. These schemata are considered to be in
competition because one and only one of them must appear in every complete
solution. An empty Bayesian Network corresponds to assuming complete
independence amongst our variables. This is equivalent to only considering
first-order schemata (schemata with exactly one $0$ or $1$); we assume that the
optimal value for any bit in our optimization problem may be found
independently of all others.

What if we add the edge $(x_i,x_j)$? We are now considering statistics
for all (four) second-order schemata involving these two variables. Assuming a
three-bit optimization problem with $i=0$ and $j=1$, where $x_2$ remains
independent, the full set of schemata we are tracking will be:
\begin{equation}
  S = \{0??, 1??, 00?, 01?, 10?, 11?, ??0, ??1\} .
\end{equation}

Note that in principle the directionality of the edge is significant; because
the edge is from $x_0$ to $x_1$, we still track the first-order statistics for
$x_0$, while those for $x_1$ are discarded in favor of the second-order
schemata listed above. If we further add the edge $(x_2,x_0)$, the schemata
considered will be:
\begin{equation}
  S = \{0?0, 0?1, 1?0, 1?1, 00?, 01?, 10?, 11?, ??0, ??1\}.
\end{equation}
If instead of $(x_2,x_0)$ we add the edge $(x_2,x_1)$, the schemata considered
will be:
\begin{equation}
  S = \{0??, 1??, 000, 001, 010, 011, 100, 101, 110, 111, ??0, ??1\}.
\end{equation}

Since we are now tracking all of the third-order schemata in a three-bit
optimization problem, the network is not doing a very good job of modeling
(i.e. compressing) the data - the Bayesian Network representation is of course
most effective when the maximal order of the schemata we need to track ($k$) is
much lower than the size of the problem ($n$), but non-zero.

So one way to look at the probabilistic models learned by the BOA is as
specifying a particular set of low-order schemata that are (hopefully)
sufficient for quickly finding a solution to our problem. If our set of
schemata is larger than strictly necessary, we will need to consider more
solutions (intuitively, the number solutions that must be considered is
$O(|S|)$). If our set of schemata is too small, however (i.e. considers
schemata of too low an order), we will only ever reach a local optimum in our
search, unless we are willing to examine on the order of $|S| \cdot n^{d}$
solutions, where $d$ is difference between the \emph{actual} order of the
largest schemata in the problem and the order of the largest schemata in $S$.

Adding edges is to a model quite expensive; if $x_j$ has $k$ incoming edges,
adding a new edge $(x_i,x_j)$ will increase the number of schemata we must
consider (i.e. the size of our conditional probabilities table) by $2^k$. To
address primarily this issue, the hierarchical BOA (hBOA)~\cite{hBOA} was
developed, which uses a finer-grained representation than a Bayesian Network -
a Bayesian Network with local structure (i.e. decision graphs or trees). The
hBOA allows us to take smaller steps in model-building, and add accordingly
fewer schemata at a time to $S$.

Consider the three-node network above, with the single edge $(x_0,x_1)$.  With
the hBOA, we can say that $x_1$ is also dependent on $x_2$, but only when
$x_0=0$. This will give us the schemata:
\begin{equation}
  S = \{0??, 1??, 000, 001, 010, 011, 10?, 11?, ??0, ??1\}.
\end{equation}
Even though the order of the model has increased from two to three, we have
only increased the number of schemata we must consider by two, not four. For
complex models, the savings from applying the hBOA vs. the BOA can be
huge. With decision graphs we can even apply \emph{merge} operations that
decrease the number of schemata maintained. If we decide that if $\{x_0=0,
x_2=1\}$ should be merged with $x_0=1$, we get the schemata:
\begin{equation}
  S = \{0??, 1??, 000, 010, ?0?, ?1?, ??0, ??1\}.
  \footnote{While merge operations aren't especially useful in practice, they
    are a nice illustration of increased expressiveness.}
\end{equation}

At this point one might ask, why bother with any kind of model structure at
all? If, at the end of the day, we are just manipulating sets of schemata, why
not just maintain such a set in memory, allowing for the finest-grained
manipulation possible? There are two significant motivations for using a
probabilistic model:

\begin{enumerate}
\item There are well-developed and tractable heuristics for learning
  probabilistic models from data.
\item Sampling from the probability distribution defined by such a model (plus
  marginals) is simple and fast.
\end{enumerate}

One can in fact design an EDA based on directly learning a set of ``useful
distinguishing schemata'' and using them to generate new instances (this is a
case of the Learnable Evolution Model approach~\cite{LEM}). But let us return
our attention to the schemata themselves.

\subsection{The Difficulty with non-Holland Schemata}

Lets consider a very simple optimization problem where the goal is to discover
a string of length $n$ that contains $k$ consecutive ones. Unfortunately, our
schema formalism is not very useful in modeling this state of affairs - to
describe the optima, we will need $n-k+1$ schemata, one per optimum. It seems
quite natural to introduce, in addition to $?$ for a single ``don't-care''
value, the symbol $*$ for ``any number of don't-care values''. The optima for
this problem may now be described with a single schema, $*1^k*$.

But can we introduce a corresponding EDA, together with model-building and
sampling procedures, for these new schemata? Unfortunately, this appears to be
rather difficult (I give it a shot in~\cite{fbBOA}). What makes modeling and
sampling so easy for EDAs based on classical schemata is the way that they
naturally carve up the search space into disjoint subsets:
\begin{enumerate}
\item First-order Holland schemata are closed under complement.
\item Holland schemata that specify values for disjoint variables may be
  specified independently of each other (e.g. $1?$ and $?0$ never conflict).
\item A set of Holland schemata may be easily sampled that fully specify a
  solution without overspecifying it.
\end{enumerate}
In contrast:
\begin{enumerate}
\item The complement of a first-order extended schemata will not usually be an
  extended schemata.
\item Extended schemata with $*$s in them will often conflict with each other.
\item The same set of extended schemata can both overspecify and underspecify
  solutions, depending on how they are applied (consider $*11*$ and $*00*$ in a
  three-bit optimization problem).
\end{enumerate}

\section{Search Procedure}

What I propose as a solution to this difficulty is a new stochastic search
procedure that may operate generically on a wide range of schema classes. We
can define a valid ``schema space''\footnote{This was Ben Goertzel's
suggestion.} a for metric solution space $X$ as a triplet
$(Y,nearest,disrupts)$ such that:

\begin{enumerate}
\item For all $x \in X$ and $y \in Y$, $nearest(x,y)$ is the set of solutions
  containing schema $y$ that are closest to $x$ (in particular $x$ contains $y$
  iff $nearest(x,y)=\{x\}$).
\item For all $Z \subseteq Y$, $disrupts(Z)$ is true iff there exists no
  solution $x \in X$ that contains every schema in $Z$.
\end{enumerate}

Mathematically, $disrupts$ is not necessary for finite solution spaces - we can
simply enumerate all solutions, and define the $disrupts$ function based on
which combinations of schemata do and do not appear in any solution. Rather, an
efficient means of determining whether or not schemata may co-occur in
solutions is necessary in order for probabilistic model-building to be
implemented efficiently and generically.

Lets assume that we have a well-chosen schema space given to us.\footnote{One
  that reduces deceptiveness and encapsulates the strongest interactions
  between variables.} How can we use this structure to approximate uniform
sampling of solutions that maximize our scoring function?  For simple problems,
and even for difficult problems such as traps which nonetheless are noise-free
and have crisp decompositions, a straightforward hill-climbing approach will be
sufficient (cf.~\cite{Iclanzan}). Let the neighbors of some solution $x$,
according to a schema space $(Y,nearest,disrupts)$, be defined as the union of
$nearest(x,y)$ over all $y \in Y$, excluding $x$. We will then do:

\begin{algorithmic}
\Procedure{Hillclimb}{Solution $x$}
  \While{termination criterion not met}
    \If{all neighbors of $x$ have been visited}
      \State \textbf{fail}
    \EndIf
    \State $x' \leftarrow$ a random neighbor of $x$
    \If{$x'$ improves on $x$}
      \State $x \leftarrow x'$
    \EndIf
  \EndWhile
\EndProcedure
\end{algorithmic}

How the schemata that define the neighborhood are acquired is not yet
specified; it may be via the BOA or hBOA, Ben Goertzel's proposed PLEASURE
(``Program Learning via Ensemble Analysis and SUbstitution of Repeated
Entities.'')  algorithm~\cite{PLEASURE}, the feature selection scheme from the
feature-based BOA~\cite{fbBOA}, etc., or some combination thereof. The way this
new procedure overcomes the difficulties of over- and under-specification is by
applying the model in a local rather than a global context; rather than
generating entire solutions \emph{de novo} according to our model, we will use
it to propose transformations to existing solutions.

Now, consider that even with the right problem decomposition, hill-climbing
will fail at noisy problems. To avoid this, we can use a metropolis sampling
scheme such as simulated annealing, where a worse move is sometimes accepted.

Furthermore, some solution spaces may have extremely large
neighborhoods. Problems with such spaces will only be tractable if will
knowledge (prior and/or acquired) of which neighbors are likely to be
promising. An examples of such a problem is Boolean maximum satisfiability,
which simple \emph{focused} random walk heuristics (instead of flipping a
random variable, flip a variable appearing in an unsatisfied clause) such as
WalkSAT~\cite{WalkSAT} can scale up to millions of variables, while simulated
annealing can only handle hundreds. 

On the other hand, these procedures do not sample optima uniformly, as
metropolis-based procedures tend to. By combining steps taken by a metropolis
approach with greedier focused moves, near-uniform sampling can be
restored~\cite{WalkSampling}.\footnote{Sampling optima near-uniformly is
  crucial for linkage learning, discussed in the next section.} Specifically,
this combination allows us to continue to explore nearby optima even after we
have already found an optimal or near-optimal solution. So our procedure
becomes the following:

\begin{algorithmic}
\Procedure{StochasticLocalSearch}{Solution $x$}
  \While{termination criterion not met}
    \If{all neighbors of $x$ have been visited}
      \State \textbf{fail}
    \ElsIf{$Metropolis()$}
      \State $x' \leftarrow$ a random neighbor of $x$
      \If{$x'$ improves on $x$ or $MetropolisNoise(x,x')$}
        \State $x \leftarrow x'$
      \EndIf
    \Else
      \State $x' \leftarrow$ a random neighbor of $x$ with highest expected
      utility
      \If{$x'$ improves on $x$ or $FocusedNoise(x,x')$}
        \State $x \leftarrow x'$
      \EndIf
    \EndIf
  \EndWhile
\EndProcedure
\end{algorithmic}

This procedure should be able to solve even noisy and difficult problems, and
sample relatively uniformly over optimal solutions when executed repeatedly
from different random initial conditions. However, on demanding problems with
very many local optima, it may still stagnate, and be unable to overcome strong
attractor basins and reach global optima. I hypothesize that this sort of
large-scale topology may still appear in a search space, irrespective of
schemata we may discover (i.e. the search space may be intrinsically
irregular). 

A proposed solution is to structure the search as a large-step Markov
chain~\cite{LSMC}. This technique is based on a paradigm originally proposed by
Baum, known as ``iterated descent''~\cite{BaumDescent}. Instead of continuing
to expend computational effort on diminishing returns after search has
stagnated, or restarting search from an entirely new location, iterated descent
proposes that a macromutation (or ``kick'') procedure be employed that
randomizes a significant number, but not all, of the subcomponents in the local
optimum that we have reached. The key to this method is for the macromutation
to be large enough to escape from the basin of attraction surrounding the
current local optimum (so that the search is unlikely to return to the same
optimum), but not so large as to destroy all of the learned structure. The
combination of a macro-mutation followed by local search is considered a single
state transition in the Markov chain, and leads to the following procedure:

\begin{algorithmic}
\Procedure{MarkovianStochasticLocalSearch}{Solution $x$}
  \While{termination criterion not met}
    \If{all neighbors of $x$ have been visited or $Stuck()$}
      \State $Macromutation()$
    \ElsIf{$Metropolis()$}
      \State $x' \leftarrow$ a random neighbor of $x$
      \If{$x'$ improves on $x$ or $MetropolisNoise(x,x')$}
        \State $x \leftarrow x'$
      \EndIf
    \Else
      \State $x' \leftarrow$ a random neighbor of $x$ with highest expected
      utility
      \If{$x'$ improves on $x$ or $FocusedNoise(x,x')$}
        \State $x \leftarrow x'$
      \EndIf
    \EndIf
  \EndWhile
\EndProcedure
\end{algorithmic}

In practice it would be desirable to use a single procedure for both
$MetropolisNoise$ and $FocusedNoise$, or to set one (or even both of them) to
zero; unfortunately there does not appear to be much theoretical guidance in
the realm of hybrid methods. 

In order to have a complete search procedure operating on generalized schemata,
we still need to specify how linkages between schemata are learned (or
equivalently, how we build probabilistic models distinguishing between good and
bad solutions); this will be described in the next section, and will take
advantage of the $disrupts$ procedure specified above as part of a schema
space.

\section{Probabilistic Model-Building}

The purpose of probabilistic model-building is to iteratively factor out any
large-scale regularities that might be present in the search space. For
tractability (these problems are at least NP-hard) I consider here only
bottom-up greedy methods (which can still be quite expensive). Specifically,
all procedures will have the general form:

\begin{algorithmic}
\Procedure{AgglomerativeModelBuilding}{Schemata $S$,Data $D$}
  \State Select eligible schemata $s_1,s_2 \in S$ most linked according to $D$
  \If{$s_1 \wedge s_2$ doesn't overfit}
    \State $S \leftarrow S \cup \{s_1 \wedge s_2\}$
  \EndIf
  \If{$\neg (s_1 \wedge s_2)$ doesn't overfit}
    \State $S \leftarrow S \cup \{\neg (s_1 \wedge s_2)\}$
  \EndIf
  \If{$S$ has changed}
    \State $S \leftarrow GreedyAgglomerativeModel(S,D)$
  \EndIf
  \State \textbf{return} $S$
\EndProcedure
\end{algorithmic}

The initial model-building call will have $S$ be the set of first-order
schemata in our schema space. Note that being able to compute whether
individual schemata are present in or absent in some solution allows us to
efficiently compute whether conjunctions and negations of schemata are present
or absent. Similarly, having the predicate $disrupts$ lets us efficiently
decide whether a pair of schemata is eligible for linkage or not, without
considering the data (i.e. we need only consider $D$ for sets of schemata $Z$
where $disrupts(Z)$ is false). For further efficiency we should exploit that
for all sets of schemata $A$ and $B$, $A \subset B \wedge disrupts(A)
\rightarrow disrupts(B)$.

When agglomerative model-building is integrated into stochastic local search,
the iterated descent process will no longer be Markovian (since grouping
together schemata as search progresses will modify transition
probabilities). Inspired by competent mutation operators that exploit
probabilistic models~\cite{CompetentMutation,CompetentMutation2}, and the
building-block hill-climber~\cite{Iclanzan,Iclanzan2}, I propose only applying
model-building to local optima or near-local optima (i.e. to solutions that
have the macromutation operator applied to them). This limits the resources
that will need to be expended on model-building, which can be quite
computationally expensive, and also reduces noise.

In order for model-building to be fully accurate up to level-$k$ interactions,
$O(n \cdot 2^k)$ ``sufficiently large'' macromutations will need to be
performed. I have not yet worked out a how to properly size macromutations, but
it should be possible along the lines of EDA population sizing equations
(cf.~\cite{DOI}). Furthermore, note that macromutations should be carried out
on a schema level (as
in~\cite{CompetentMutation,CompetentMutation2,Iclanzan2}), rather than on the
level of individual variables. Doing so efficiently requires using the
$disrupts$ relationship. A simple heuristic for this is:
\begin{algorithmic}
\Procedure{Macromutate}{Schemata $S$, Solution $x$,Size $size$}
  \State $z \leftarrow$ a random schema from $S$
  \State remove $z$ from $S$
  \While{$|Z|<size$ and $|S|>0$}
    \State $s \leftarrow$ a random schema from $S$
    \If{$\neg disrupts(\{s,z\})$}
      \State $z \leftarrow z \wedge \{s\})$
    \EndIf
    \State remove $s$ from $S$
    \State $x \leftarrow$ a random solution from $nearest(x,z)$
  \EndWhile
  \State \textbf{return} $x$
\EndProcedure
\end{algorithmic}
This procedure assumes $S$ contains only schemata not present in $x$:

Finally, note that for both agglomerative model building and macromutation, I
am assuming that our schema space is closed under conjunction. If one wants to
learn negated schemata in agglomerative model-building, it must be closed under
negation as well.

\section{Assigning and Updating Utilities}

I will assume that the values returned by the scoring function are linearly
scaled - e.g. an improving the score of the current solution by $c$ at one
point in the search is as good as improving the score by $c$ some other point
in the search, improving the score of the current solution by $2c$ is as good
as improving it by $c$ twice at other points in the search, etc.\footnote{If it
  turns out empirically to be the case that progressing via fewer large jumps
  is less computationally costly than progressing via many small jumps, or vice
  versa, this preference may be incorporated as well, though it is not obvious
  to me that this distinction is worth pursuing.}  On this basis, given a
solution $x$, a set of schemata $S$ that are not present in $x$, a set $X$ of
solutions that have been visited, and a scoring function $f$, the set of
solutions with highest expected utility may be computed as:
\begin{align}
nearest(\argmax_{s \in S} \qquad &E(f(x')|x' \in X, present(x',s)) \nonumber \\
                              - &E(f(x')|x' \in X, \neg present(x',s))) .
\end{align}

It may be necessary for performance concerns to use an approximation to this
score. The need arises from two issues:

\begin{enumerate}
\item There may be too many visited solutions to keep track of them all and
  which schemata they contain.
\item There may be too many schemata for us to do an $\argmax$ operation over
  all of them.
\end{enumerate}

The first of these issues may be address via any number of techniques from the
evolutionary computation literature, together with some clever caching. To
address the second I propose that the set of schemata $S$ be mapped onto a
directed acyclic graph, such that children are mapped to subsets of their
parents, and leaves are mapped to individual schemata in $S$. When we don't
have the resources to select the highest-utility move over the entire tree, we
can use the approximation of greedily traversing the DAG from the root downward
until we reach a leaf. There are still details to be worked out here...

It is additionally worth noting that prior probabilities and structural
knowledge regarding relationships between schemata (e.g. that a certain set of
schemata form a class of similar moves) can and should be incorporated into
expected utility computations as well.

\section{Phenotypic Schemata}

Let's take a step back for a second and take a broader view of optimization
processes. Schemata such as we have considered in a schema space may be
considered to be ``genotypic'' (intensional) - they correspond to known
properties of solutions (we can easily determine solutions with these
properties via the $nearest$ function) with known interrelationships
(determined via the $disrupts$ predicate). There are also ``phenotypic''
(extensional) properties of solutions, that may vary from problem to
problem. For example, in a supervised categorization problem, a solution's
phenotype might correspond to a vector containing its results on all training
data.

There are two ways this additional information might be used:
\begin{enumerate}
\item Under some circumstances it might be desirable, when reaching a stuck
  condition, to perform backtracking to an earlier point in the search rather
  than to perform a macromutation. Keeping track of a set of phenotypically
  diverse (specifically, non-dominating~\cite{pairwisecomp}) solutions seems a
  useful heuristic for identifying backtracking points.
\item More ambitiously, we could do learning over which phenotypic features are
  important (possibly building a model of them via a cheap technique such as
  linear regression), and do alternating searches on
  good-phenotypic-feature-maximization vs. the actual scoring function (as in
  the STAGE approach~\cite{stage}).
\end{enumerate}

Similarly, in cases where we have multiple processors and wish to carry out
search in parallel, diversity of phenotypic schemata seems like a good
heuristic. Alternatively, we could simply apply $n$ different macromutation
operations to provide starting points for $n$ different processors, although
for some problems this might lead to duplication of effort.

\section{Summary and Open Questions}

I have described an adaptive large-step chain EDAs with a number of attractive
properties:
\begin{enumerate}
\item It is a generalization simulated annealing.
\item It is a generalization of breadth-first search.
\item It is a generalization of best-first search.
\item It can operate on any type of schemata (or combination of types) - all
  that we need is a predicate indicating whether a given schema is present or
  absent in a solution, a predicate indicating whether a set of schemata may be
  applied in conjunction to a given solution, and a procedure for transforming
  a solution to contain a set of schemata that are not already present.
\item Under-specification is not a problem because we are basing our
  transformations on a solution that is already fully specified.
\item Over-specification is not a problem because we explicitly remove from
  consideration those combinations of schemata than conflict.
\item If we consider schemata that are negations of features, we can also learn
  from the commonalities in \emph{bad} solutions that have been observed (as
  proposed in PLEASURE~\cite{PLEASURE}, for instance).
\item It supports soft chunking by allowing conflicting schemata at different
  scales to be considered (e.g., $1?001??$ together with $??00???$). 
\item Relatedly, it facilitates further movement in the direction of
  fine-grained search - just as hBOA allows us to take smaller steps than BOA,
  the new approach allows us to make even smaller modifications to our model.
\item It introduces an important \emph{level of indirection} - search does not
  act directly on solution space, but views and transforms solutions only via
  schemata that are deemed useful, based potentially on both prior knowledge
  and on learning.
\end{enumerate}

There are certainly a number of questions with this approach that will need to
be resolved through further theory and experiment, including:
\begin{enumerate}
\item What is the best combination (robust across many problems of interest) of
  noise, macromutation, and/or backtracking?
\item If noise is used, how to tune it, if a constant value is insufficient? A
  simple proposal from the local search literature is adapt noise dynamically
  based on ratio of the mean and the variance of the scores of the solutions
  uncovered~\cite{ls_invariants}. A value 10\% above the minimum is proposed. A
  more advanced system might be based on thermodynamic simulated
  annealing~\cite{tsa}.
\item If macromutation is used, how to best set and adapt the macromutation
  size?
\item How to best learn linkages? Adapting the Bayes-Dirichlet metric (as used
  in the hBOA~\cite{hBOA}), or even something simpler, seems like a decent
  approach.
\end{enumerate}

\section{Acknowledgements}

Thanks to Ben Goertzel, David Iclanzan, and Martin Pelikan for feedback on
early drafts of this document.

\bibliographystyle{plain}
\bibliography{refs}
\end{document}
